% ****************************************************************************************
% ************************     	NAME OF DOCUMENT	  ************************************
% ****************************************************************************************


% =======================================================
% =======	ALL COMMANDS AND RULES FOR DOC 	 ============
% =======================================================
\documentclass[12pt]{article}							    %Type of docuemtn and size of font
\usepackage[margin=1.2in]{geometry}							%Margins

\usepackage[spanish]{babel}									%Please use spanish
\usepackage[utf8]{inputenc}									%Please use spanish	
\usepackage[T1]{fontenc}									%Please use spanish

\usepackage{amsthm, amssymb, amsfonts}				        %Make math beautiful
\usepackage[fleqn]{amsmath}                                 %Please make equations left
\decimalpoint												%Make math beautiful
\setlength{\parindent}{0pt}									%Eliminate ugly indentation

\usepackage{graphicx}										%Allow to create graphics
\usepackage{wrapfig}                                    	%Allow to create images
\graphicspath{ {Graphics/} }                                %Where are the images :D
\usepackage{listings}										%We will be using code here
\usepackage[inline]{enumitem}								%We will need to enumarate

\usepackage{fancyhdr}										%Lets make awesome headers/footers
\renewcommand{\footrulewidth}{0.5pt}						%We will need this!
\setlength{\headheight}{16pt} 								%We will need this!
\setlength{\parskip}{0.5em}									%We will need this!
\pagestyle{fancy}											%Lets make awesome headers/footers
\lhead{\footnotesize{\leftmark}}							%Headers!
\rhead{\footnotesize{\rightmark}}							%Headers!
\lfoot{Compilando Conocimiento}								%Footers!
\rfoot{Oscar Rosas}						                    %Footers!

\author{Oscar Andrés Rosas}						            %Who I am




% =====================================================
% ============     	  COVER PAGE	   ================
% =====================================================
\begin{document}
\begin{titlepage}

	\center
	% ============ UNIVERSITY NAME AND DATA =========
	\textbf{\textsc{\Large Proyecto Compilando Conocimiento}}\\[1.0cm] 
	\textsc{\Large Calculo}\\[1.0cm] 

	% ============ NAME OF THE DOCUMENT  ============
	\rule{\linewidth}{0.5mm} \\[1.0cm]
		{ \huge \bfseries Calculo Integral}\\[1.0cm] 
	\rule{\linewidth}{0.5mm} \\[2.0cm]
	
	% ====== SEMI TITLE ==========
	{\LARGE Métodos de Integración e Integrales en General}\\[7cm] 
	
	% ============  MY INFORMATION  =================
	\begin{center} \large
	\textbf{\textsc{Autor:}}\\
	Rosas Hernandez Oscar Andres
	\end{center}

	\vfill

\end{titlepage}




% ======================================================================================
% ==================================     DOCUMENT  =====================================
% ======================================================================================


% =====================================================
% ============    INTEGRACION IMPROPIA       ==========
% =====================================================
\section{Integrales Impropias}

% ==========================
% ==== EXPLICACION =========
% ==========================
\subsection{¿Qué son?}

Al definir la integral definida $\int_a^b f(x) dx$ estamos hablando de una función en la que:

\begin{itemize}
    \item Esta definida en ese intervalo.
    \item No tiene una discontinuidad infinita
    \item Obviamente el intervalo es finito
\end{itemize}

Pero, que pasaría si no fuera así...

Las integrales impropias explorar esta posibilidad así que veasmola:

% ======= TIPO 1 ========
\subsubsection{TIPO 1) Intervalos Infinitos: Sumando de Verdad}

Si la $\int_a^t f(x) dx$ existe para todo número $t \geq a$, entonces lo siguiente es verdad, siempre que exista el límite (como un número finito).
\begin{equation}
    \int_a^{\infty} f(x) dx = \lim_{t \to \infty} \int_a^t f(x) dx
\end{equation}


Si la $\int_t^b f(x) dx$ existe para todo número $b \leq t$, entonces lo siguiente es verdad, siempre que exista el límite (como un número finito).
\begin{equation}
    \int_{- \infty}^b f(x) dx = \lim_{t \to - \infty} \int_t^b f(x) dx
\end{equation}


Las integrales impropias $\int_a^{\infty}f(x)dx$ y esta $\int_{-\infty}^bf(x)dx$ se llaman \textbf{convegentes} si el límite existe y  \textbf{divergente} sino.

Si $\int_a^{\infty}f(x)dx$ y $\int_{-\infty}^bf(x)dx$ son convergentes, entonces se define esta asombrosa integral como (donde a es cualquier número que tu quieras ;) ):
\begin{equation}
    \int_{-\infty}^{\infty} f(x) dx = \int_{-\infty}^{a} f(x) dx + \int_{a}^{\infty} f(x) dx   
\end{equation}


\subsection{Ejemplo}
Podemos ver que con lo que sabemos ya podemos calcular la siguiente integral:

\begin{equation*}
    \int_1^{\infty} \frac{1}{x^2} dx = \lim_{t \to \infty} \int_1^t \frac{1}{x^2} dx  =
    \lim_{t \to \infty}  \frac{-1}{t} \rvert_{1}^{t} = \lim_{t \to \infty}  1 - \frac{1}{t} = 1
\end{equation*}


\clearpage
% ======= TIPO 2 ========
\subsubsection{TIPO 2) Funciones Discontinuas}

Si $f(x)$ es continua en $[a, b)$  pero discontinua en b, entonces (si el límite existe y es finito):
\begin{equation}
    \int_a^b f(x) dx = \lim_{t \to b^-} \int_a^t f(x) dx
\end{equation}

Si $f(x)$ es continua en $(a, b]$  pero discontinua en a, entonces (si el límite existe y es finito):
\begin{equation}
    \int_a^b f(x) dx = \lim_{t \to a^+} \int_t^b f(x) dx
\end{equation}



Si $\int_a^bf(x)dx$ es convergente, entonces se define esta asombrosa integral como (donde $c$ es $a<c<b$ ):
\begin{equation}
    \int_a^b f(x) dx = \int_a^c f(x) dx + \int_c^b f(x) dx  
\end{equation}



% =====================================================
% ============        BIBLIOGRAPHY   ==================
% =====================================================
\clearpage
\bibliographystyle{plain}
	\begin{thebibliography}{9}

	% ============ REFERENCE #1 ========
	\bibitem{Sitio1} 
		ProbRob
		\\\texttt{Youtube.com}


	 

\end{thebibliography}



\end{document}